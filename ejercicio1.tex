\documentclass{article}
\usepackage[utf8]{inputenc}
\usepackage[spanish,mexico]{babel}
\usepackage{mathtools}
\usepackage{amsmath}
\usepackage{enumerate}

\begin{document}

\begin{itemize}

  \item[P1.1] Respuesta al escalón de un sistema de primer orden
Calcular y graficar la solución del sistema (1.4) considerando un volumen inincial $ V(0)=0 $ y que el caudal de entrada es una función escalón unitario, es decir, $Q(t)=\epsilon(t)$, donde 
\begin{equation}
\epsilon(t) = \left\{ \begin{array}{rl}
  0 &\mbox{ si $x(t) < 0$} \\
  1 &\mbox{ si $x(t) \leq 0$ }
       \end{array} \right .
 \label{P1.1a} \tag{P1.1a}
\end{equation}
Utilizar los parámetros: $\rho = 1$, $g=9.8$, $A= 1$ y $R = 1$.

  \item[P1.2] Solucion libre de un sistema de segundo orden
	Para el sistema (1.5):
	\begin{enumerate}[a)]
	\item Reescribir la ecuación de forma matricial
	\item Calcular y graficar la solución con condiciones iniciales $c_{e}(0)=1$, $c_{s}(0)=0$, considerando los parámetros $r_{a}=2$ y $r_{e}=1$
	\item Repetir el punto anterior los parámetros $r_{a}=20$, $r_{e}=1$ y para $r_{a} = 0.1$, $r_{e}=1$.
	\end{enumerate}
  \item[P1.3] Respuesta al escalón de un sistema de segundo orden para el sistema masa resorte (1.7):
	\begin{enumerate}[a)]
	\item Reescribir la ecuación en forma matricial.
	\item Calcular y graficar la solución considerando condiciones iniciales nulas y que la fuerza de entrada es un escalón unitario, $F(t)=\epsilon(t)$ definido en la Ec(P1.1a). Suponer que los parámetros valen $m=k=b=1$.
	\end{enumerate}

  \item[P1.4] Transformación de una DAE en una ODE explícita. Convertir la DAE (1.9)-(1.10) en una ecuación diferencial ordinaria y escribirla en forma matricial.

  \item[P1.5] Estabilidad de los sitemas lineales y estacionarios. Analizar la estabilidad de los sistemas de los Ejemplos 1.1, 1.3, 1.4, 1.5 y 1.7. Considerar en todos los casos que todos los parametros son positivos.

  \item[P1.6] Punto de equilibrio de un sistema lineal y estacionario. Para el sistema hidráulico (1.4), considerando $Q(t)=\bar{Q}$ (constante) y los parámetros utilizandos en el Problema P1.1, calcular el \textit{punto de equilibrio}. Es decir, se pide cualcular el valor del estado (en este caso $V(t)$) para el cual la solución queda en un estado estacionario ($\dot{V}(t)=0$)
Corroborar luego este resultado con la solución del sistema calculada en el Problema P1.1.

   \item[P1.7] Punto de equilibrio en sistemas no lineales Calcular el/los puntos/s de equilibrio de los sistemas:
	\begin{enumerate}[a)]
	\item El sistema (1.3), considerando $b=m=1$, $g=9.8$
	\item El sistema presa-depredador (1.8), suponiendo $r=a=b=m=0.1$. 
	\end{enumerate}

Interpretar los resultados.




   \item[P1.8] Pelota rebotando. El siguiente modelo puede verse como una combinación del sistema (1.3) (suponiendo rozamiento lineal y con $v(t)$ positiva hacia arriba) y el sistema (1.7) (con $F(t)=-m.g$):

\begin{align*}
  \dot{x}(t) &= v(t) \\
 \label{P1.8a} \tag{P1.8a}
  \dot{v}(t) &=   \begin{cases}
    -\frac{b_{a}}{m} \cdot v(t) - g & \text{si } x(t) > 0\\
    -\frac{k}{m} \cdot x(t) - \frac{b}{m} \cdot v(t) - g & \text {si} x \leq 0,
  \end{cases}\\
\end{align*}
\end{itemize}

\end{document}
